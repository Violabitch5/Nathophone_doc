\subsection{Funktionstest Driverboard}
Nach Eintreffen des Driverboards wurde das Board zuerst optisch kontrolliert. Leider war SW2 nicht bestückt, was allerdings die Funktion nicht direkt beeinträchtigte.\\Zudem waren die I2C-Pullup Widerstände R23 und R24 bestückt, obwohl sie eigentlich nicht bestückt sein sollten. Dies war kein so grosses Problem im Standalone-Betrieb, jedoch würden mit dem Milan-Modul zusammen mehrere Pullups parallel geschaltet werden, was zu einem zu grossen Stromfluss auf den I2C-Leitungen führen könnte. Also mussten diese Widerstände vor Betrieb mit dem Modul zuerst entfernt werden.
\subsubsection{Aufstartverhalten}Es wurde als Speisung ein Netzteil mit +24V und 0.5A Strombegrenzung eingestellt und an die Speisespannung (Main Power IN) angelegt. Dabei wurde beobachtet, dass die Stromaufnahme sofort 0.5A betrug, ohne dass überhaupt eine Last angehängt war. Die Ursache dafür wurde gefunden, als bemerkt wurde, dass dieser Strom auch floss wenn der Hauptschalter (SW1) noch gar nicht umgelegt war: Die TVS-Diode schien bereits im Durchbruchbereich zu sein.\\Ein genauerer Blick in das Datenblatt zeigte, dass diese TVS-Diode tatsächlich ein asymetrisches Modell war, also unterschiedliche Standoff- und Durchbruchsspannungen hatte je nach Polarisierung (siehe Abblidung \ref{pic:datasheet_TPSMB2616CA})\footnote{Da die Bauteile in der Regel parametrisch gesucht wurden, war diese Eigenschaft bei der tabellarischen Anzeige nicht sichtbar. Auch auf der Produkteseite war nichts dahingehend vermerkt.}. Im Gegensatz zu einer symetrischen TVS-Diode war es also sehr wichtig, in welche Richtung das Bauteil ausgerichtet war. Leider kam die Ausrichtung genau so zu liegen, dass die K-Seite auf GND verbunden war, also die Standoff-Spannung bei 16V lag und daher die Diode bereits hier im Durchbruch lag.
\begin{figure}[H]
	\centering
	\includegraphics[width=\textwidth - 2cm]{pictures/TPSMB2616CA_breakdowndata.png}
	\caption{Die Breakdown-Daten der TPSMB2616CA}
	\label{pic:datasheet_TPSMB2616CA}
\end{figure}
Da dieses Bauteil mit relativ grossen Kupferflächen verbunden war, musste spezielles Werkzeug verwendet werden, um das Bauteil zu entlöten und zu drehen.
\subsubsection{I2C-Verhalten}Wie in \ref{i2c_device_status_registers} beschrieben, besitzt jeder Endstufen-Chip ein Device-ID Register, welches schlichtweg mit 0x01 beschrieben ist. Dieses konnte erfolgreich aus allen sechs Chips ausgelesen werden, auch ohne angeschlossenes Milan-Modul. Dies bedeutete, dass die 3.3V-Speisung stabil arbeitet, also auch die 5V-Speisung stabil arbeitet.