Anschliessend wurden die Ziele jeweils gegeneinander verglichen und eines davon als Präferenz ausgewählt. Somit entstand eine in Tabelle \ref{tab:aimweights} ersichtliche Präferenzmatrix. Aus den Anzahl gewonnenen Vergleichen kann nun pro Ziel ein Rang und eine Gewichtung ermittelt werden.
\begin{table}[!ht]
	\centering
	%\rowcolors{2}{gray!0}{gray!8}
	\begin{tabularx}{\textwidth}{cccclccccc}
		\toprule
		\textbf{Rang} & \textbf{Gewicht} & \textbf{Anzahl} & \multicolumn{2}{c}{\textbf{Zielbezeichnung}} & \multicolumn{4}{c}{\textbf{Präferenzen}} \\ \midrule
		\textbf{2} & 18 & 4 & \textbf{A} & Direktionale Abstrahlung & ~ & ~ & ~ & ~ & ~ \\ 
		\textbf{} & ~ & ~ & ~ & ~ & A & ~ & ~ & ~ & ~ \\ 
		7 & 5 & 1 & \textbf{B} & Möglichst wenige Stecker & ~ & A & ~ & ~ & ~ \\ 
		\textbf{} & ~ & ~ & ~ & ~ & B & ~ & A & ~ & ~ \\ 
		9 & 0 & 0 & \textbf{C} & Untere Grenzfrequenz tief genug & ~ & D.1 & ~ & ~ & ~ \\ 
		\textbf{} & ~ & ~ & ~ & ~ & D.1 & ~ & ~ & ~ & ~ \\ 
		4 & 9 & 2 & \textbf{D.1} & Mobilität (Gewicht) & ~ & ~ & ~ & A & ~ \\ 
		\textbf{} & ~ & ~ & ~ & ~ & D.2 & ~ & ~ & ~ & ~ \\ 
		\textbf{2} & 18 & 4 & \textbf{D.2} & Mobilität (Dimensionen) & ~ & D.2 & ~ & ~ & ~ \\ 
		\textbf{} & ~ & ~ & ~ & ~ & D.2 & ~ & D.2 & ~ & G \\ 
		9 & 0 & 0 & \textbf{E} & Speisung+Daten auf einem Stecker & ~ & F & ~ & ~ & ~ \\ 
		\textbf{} & ~ & ~ & ~ & ~ & F & ~ & ~ & ~ & ~ \\ 
		4 & 9 & 2 & \textbf{F} & Abstrahlung softwaremässig steuerbar & ~ & ~ & ~ & G & ~ \\ 
		\textbf{} & ~ & ~ & ~ & ~ & G & ~ & ~ & ~ & ~ \\ 
		\textbf{1} & 27 & 6 & \textbf{G} & Reduziertes Brandrisiko & ~ & G & ~ & ~ & ~ \\ 
		\textbf{} & ~ & ~ & ~ & ~ & G & ~ & G & ~ & ~ \\ 
		4 & 9 & 2 & \textbf{H} & Benutzersicherheit & ~ & ~ & ~ & ~ & ~ \\ 
		\textbf{} & ~ & ~ & ~ & ~ & H & ~ & ~ & ~ & ~ \\ 
		9 & 0 & 0 & \textbf{I} &  Batteriebetrieben & ~ & H & ~ & ~ & ~ \\ 
		\textbf{} & ~ & ~ & ~ & ~ & J & ~ & ~ & ~ & ~ \\ 
		7 & 5 & 1 & \textbf{J} &  Drahtlose Signalübertragung & ~ & ~ & ~ & ~ & ~ \\ 
		\bottomrule
	\end{tabularx}
	\caption{Zielgewichtung}
	\label{tab:zielgewichtung}
\end{table}
Somit konnten die drei Hauptziele eruiert werden:
\paragraph{G Reduziertes Brandrisiko} Das Endprodukt muss ein möglichst minimiertes Brennbarkeitsrisiko aufweisen. Zwar gibt es mit der EN 13501-1 eine Klassifikation zum Brandverhalten, jedoch behandelt dieses rein Baustoffe und nicht ein Produkt als ganzes. Dieses Ziel ist insbesondere relevant, da u.U. Brennbare Materialen wie MDF in Kombination von Leistungsendstufen vorkommen.
\paragraph{D.2 Mobilität}
\paragraph{A Direktionale Abstrahlung}