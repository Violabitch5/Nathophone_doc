Aus den beschriebenen Zielen können einige Widersprüche festgestellt werden:
\paragraph{C vs. D.2} Eine tiefe untere (akustische) Grenzfrequenz bedingt immer ein grösseres Volumen des Resonanzkörpers. Eine Mobilität bedingt eine gewisse \textit{Handlichkeit} bzw. Transportfähigkeit. Somit setzten sich diese Ziele direkt im Widerspruch.
\paragraph{I vs. G} Je nach Batterietyp können diese sehr wohl ein Risiko der Brennbarkeit bergen. Somit ist mit einer Batterie automatisch die Brennbarkeit erhöht.
Allerdings ergab sich auch ein sich ergänzendes Zielpaar:
\paragraph{B vs. E} Beide Ziele haben im Endeffekt den selben Fokus. Eine ideale Erfüllung wäre hier ein einziger Stecker mit Speisung und Datensignalen.