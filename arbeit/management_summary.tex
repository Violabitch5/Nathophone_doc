\subsection*{Projektübersicht}
Diese Diplomarbeit befasst sich mit der Entwicklung und Herstellung eines \textit{Beam Steering}-fähigen Line-Array-Systems. Die Grundlage bildete die Frage, ob ein \textit{Beam Steering}-fähiges Saiteninstrument entwickelt und hergestellt werden konnte, welches Schall gezielt in eine bestimmte Richtung abstrahlen kann.\\Das Projekt umfasste die komplette Entwicklung von der Konzeptphase über das System Engineering, Simulationen, Bauteileevaluation, PCB-Design bis hin zur Herstellung und Funktionstests. Der Fokus lag dabei auf einer praktischen Umsetzung innerhalb von 14 Wochen. Bereits früh in der Projektphase zeigte sich, dass eine Umsetzung mittels einer Saite enorm umständlich und einige Risiken bot. Daher musste das namensgebende Konzept abgeändert werden, da fix-fertig verfügbare Exciter eingesetzt wurden. Leider konnte die Inbetriebnahme des kompletten Systems wegen Lieferverzögerungen eines MILAN-Moduls nicht mehr innerhalb der gegebenen Zeit realisiert werden.
\subsection*{Methodisches Vorgehen}
Die Entwicklung begann mit dem System Engineering. Nach einer SWOT- und Ishikawa-Analyse wurden zehn Projektziele definiert und gewichtet. Als Primärziele ergaben sich daraus: \textbf{Reduziertes Brandrisiko}, \textbf{Mobilität bezüglich Dimensionen}, \textbf{Direktionale Abstrahlung}\\Mittels eines morphologischen Kastens wurden nun fünf Systemvarianten generiert und anhand einer Nutzwertanalyse sowie Kosten-Nutzen-Analyse bewertet. Die Variante \textquotedbl{}Einfache Anwendung, Plug'n'Play\textquotedbl{} erwies sich als optimal und wurde zur Umsetzung ausgewählt.
\subsection*{Umsetzung}
Das entwickelte System basierte auf folgenden Hauptkomponenten:
\begin{itemize}
	\item Sechs DAEX25-Exciter auf Plexiglasplatten.
	\item Das entwickelte Driverboard enthielt sechs TAS5720 Endstufen. Die Speisungen wurden mit einem LT8650S Dual-Output Schaltregler erzeugt.
	\item Ein MILAN-Modul der Firma JOYNED. Dies erfüllte das Ziel einer Plug'n'Play-Lösung mit minimaler Steckerzahl.
\end{itemize}
\subsection*{Herstellung und Fazit}
Das Gehäuse wurde mittels CO2-Lasercutter des Fablab Winterthur aus 8mm MDF in acht Schichten hergestellt und zusammengeleimt. Die PCB-Fertigung und Bestückung erfolgte bei AISLER.\\Das Projekt erreichte trotz beträchtlicher Lieferverzögerung des MILAN-Moduls einen zufriedenstellenden Stand. Alle Hardwarekomponenten wurden erfolgreich entwickelt, hergestellt und so weit es ging getestet. Das PCB zeigte nach einigen kleineren Anpassung ein einwandfreies Verhalten. Das MILAN-Modul konnte jedoch nicht mehr im Rahmen dieser Arbeit integriert werden.