Das Grundlegende Projektziel ist eigentlich sehr einfach beschrieben: Es soll ein Gerät entwickelt werden, welches den Klang in eine bestimmte Richtung abstrahlen kann (Beam Forming). Da jedoch in Rahmen dieser Ausbildung bzw. dieses Projekts keine anechoische Kammer und/oder eine genau winkelverstellbare Halterung zur Verfügung stand, um die Direktionalität des Schallpegels zu messen, war die grundsätzliche Messbarkeit dieses Zieles fraglich.\\Nichtsdestotrotz sollen sowohl das oben genannte Hauptziel (\textbf{Muss}) als auch Nebenziele (\textbf{Soll, Kann}) an dieser Stelle definiert und quantifiziert werden. Dabei ist der Ziel-Zeitpunkt jeweils der Abgabetermin der Arbeit. Tabelle \ref{tab:aims} zeigt die verschiedenen Ziele und deren Messgrössen.\\
\begin{table}[H]
	\centering
	%\rowcolors{2}{gray!0}{gray!8}
	\setstretch{1.5}
	\begin{tabularx}{\textwidth}{l|>{\columncolor{red!22}}c|>{\columncolor{orange!12}}c|>{\columncolor{blue!12}}c|ccc}
		\centering
		Zielbezeichnung & \textbf{MUSS} & \textbf{SOLL} & \textbf{KANN} & Messgrösse & Kenn/Grenzzahl & Bedingung  \\
		\hline
		{\Large \textbf A}\hspace{3mm}\begin{minipage}{2.4cm}
			\vspace{2mm}
			\setstretch{1.1}
			Direktionale Abstrahlung
			\vspace{2mm}
		\end{minipage} & x &  &  & Direktivität & -3dB SPL & \begin{minipage}{2.1cm}
			\vspace{1mm}
			\setstretch{1.1}
			\centering
			> 10° von Bezugsachse
		\end{minipage}\\
		\hline
		{\Large \textbf B}\hspace{3mm}\begin{minipage}{2.4cm}
			\vspace{2mm}
			\setstretch{1.1}
			Möglichst wenige Stecker
			\vspace{2mm}
		\end{minipage} &  & x &  & Anzahl Stecker & max. 3 & - \\
		\hline
		{\Large \textbf C}\hspace{3mm}\begin{minipage}{2.4cm}
			\vspace{2mm}
			\setstretch{1.1}
			Untere Grenzfrequenz tief genug
			\vspace{2mm}
		\end{minipage} &  & x &  & -3dB Punkt & min. 100 Hz & - \\
		\hline
		{\Large \textbf D.1}\hspace{1.5mm}\begin{minipage}{2.4cm}
			\vspace{2mm}
			\setstretch{1.1}
			Mobilität
			\vspace{2mm}
		\end{minipage} &  & x &  & Gewicht & max. 6 kg & - \\
		%\hline
		{\Large \textbf D.2}\hspace{1mm}\begin{minipage}{2.4cm}
			\vspace{2mm}
			\setstretch{1.1}
			Mobilität
			\vspace{2mm}
		\end{minipage} &  &  & x & Dimensionen & max. 1.8x0.8x0.3m & - \\
		\hline
		{\Large \textbf E}\hspace{3mm}\begin{minipage}{2.4cm}
			\vspace{2mm}
			\setstretch{1.1}
			Speisung + Daten auf einem Stecker
			\vspace{2mm}
		\end{minipage} &  &  & x & Anzahl Stecker & 1 & - \\
		\hline
		{\Large \textbf F}\hspace{3mm}\begin{minipage}{2.4cm}
			\vspace{2mm}
			\setstretch{1.1}
			Abstrahlung softwaremässig steuerbar
			\vspace{2mm}
		\end{minipage} &  &  & x & Möglich & Ja & -\\
		\hline
		{\Large \textbf G}\hspace{3mm}\begin{minipage}{2.4cm}
		\vspace{2mm}
		\setstretch{1.1}
		Reduziertes Brandrisiko
		\vspace{2mm}
		\end{minipage} & x &  &  & \begin{minipage}{2cm}
			\centering
			\vspace{2mm}
			MTBF (durch Brand)\vspace{2mm}
		\end{minipage} & min. 200 Jahre & \begin{minipage}{2.4cm}
		\vspace{1mm}
		\setstretch{1.1}
		\centering
		Sachgemässe Benützung
		\end{minipage}\\
		\hline
		{\Large \textbf H}\hspace{3mm}\begin{minipage}{2.4cm}
		\vspace{2mm}
		\setstretch{1.1}
		Benutzersicherheit
		\vspace{2mm}
		\end{minipage} &  & x &  & \begin{minipage}{2cm}
		\centering
		\vspace{2mm}
		MTBF (durch Benutzerunfall)
		\vspace{2mm}
		\end{minipage} & min. 80 Jahre & \begin{minipage}{2.4cm}
		\vspace{1mm}
		\setstretch{1.1}
		\centering
		Sachgemässe Benützung
		\end{minipage} \\
		\hline
		{\Large \textbf I}\hspace{3mm}\begin{minipage}{2.4cm}
		\vspace{2mm}
		\setstretch{1.1}
		Batteriebetrieben
		\vspace{2mm}
		\end{minipage} &  &  & x & Akkulaufzeit & 12h & - \\
		\hline
		{\Large \textbf J}\hspace{3mm}\begin{minipage}{2.4cm}
		\vspace{2mm}
		\setstretch{1.1}
		Drahtlose Signalübertragung
		\vspace{2mm}
		\end{minipage} &  &  & x & Funkstrecke & Ja & -
	\end{tabularx}
	\label{tab:aims}
	\caption{Projektziele}
\end{table}