\subsection{Simulation des Schaltreglers}
Auf der Produktewebseite des LT8650S\footnote{Siehe: \href{https://www.analog.com/en/products/lt8650s.html}{Analog Devices - LT8650S}} konnte ein vorbereitetes Simulations-File für LTSpice heruntergeladen. Dieses diente als Ausgangslage, um die schlussendliche Speisungsschaltung zu simulieren. Abbildung \ref{pic:example_simulation_LT8650S} zeigt den Aufbau dieser Beispiel-Simulation und entspricht der Konfiguration mit 5V- und 3.3V-Ausgängen.
\begin{figure}[H]
	\centering
	\includegraphics[trim={0 7cm 0 7cm},clip,width=\textwidth*3/4]{pictures/LT8650S_example_simulation.pdf}
	\caption{Aufbau der Beispielsimulation des LT8650S}
	\label{pic:example_simulation_LT8650S}
\end{figure}
Dieser Aufbau wurde nun auf die vorliegende Anwendung angepasst. Schritt für Schritt konnte und die Schaltung optimiert werden. Dabei wurde die Last allerdings als rein statischer DC-Strom angenommen, welcher von der ersten $\mu s$ an den vollen Strom bezieht. Dies ist nicht wirklich realitätsnah, aber gibt u.U. Hinweise auf das Verhalten bei solchen Extremsituationen. Abbildung \ref{pic:adapted_simulation_LT8650S} zeigt die adaptierte Simulation.
\begin{figure}[H]
	\centering
	\includegraphics[trim={0 7cm 0 7cm},clip,width=\textwidth*3/4]{pictures/LT8650S_adapted_simulation.pdf}
	\caption{Adaptierte Simulation des LT8650S}
	\label{pic:adapted_simulation_LT8650S}
\end{figure}
Zu bemerken ist, dass die Schaltfrequenz $f_{sw}$ bei 2MHz beibehalten wurde. Die sonstigen Anpassung waren wie folgt:
\subsubsection{OUT2 auf 15V}
Die erste Anpassung war logischerweise die Anpassung der Ausgangsspannung des zweiten Outputs. Dazu musste lediglich \textbf{R4 auf 56k}\footnote{Die entsprechende Formel aus dem Datenblatt: $R1 = R2 (\frac{V_{OUT}}{0.8V}-1)$, mit R1 zwischen FB und Output, und R2 zwischen FB und GND.} gesetzt werden.
\subsubsection{Lastentkopplung durch drei Ls}
Da die Endstufen jeweils durch 150uH-Induktivitäten und 400uF-Kapazitäten entkoppelt werden, wurden diese entsprechend eingebunden.
\subsubsection{Erhöhung der Induktivität L2}
Als zentrales Bauteil eines jeden Schaltreglers fungiert die Ausgangsspule. Das Datenblatt gibt einen Anhaltspunkt, wie gross die Induktivität gewählt werden sollte:
\begin{quote}
	A good first choice for the inductor value is:\\
	$L_{1,2} = \frac{V_{OUT1,2} + V_{SW(BOT)}}{2f_{SW}}$\\
	where fSW is the switching frequency in MHz, VOUT is the output voltage, VSW(BOT) is the bottom switch drop (~0.12V) and L is the inductor value in μH.
	- \cite{LT8650S_Datasheet}
\end{quote}
Interessant dabei ist, dass die Einheiten ($\frac{V}{Hz} = Vs$) hier keine Induktivität als Einheit ($\frac{Vs}{A}$) entsteht. Mit den hier vorhandenen Werten eingesetzt entspräche das $L_{2} = \frac{15V + 0.12V}{2\cdot2MHz} = 3.78uH(?)$. Dies ist wohl eher als Faustregel zu verstehen.
Eine andere Formel aus dem Datenblatt gibt weiteren Aufschluss über das Verhalten. Diese beschreibt den Rippelstrom ($\Delta I_{L}$) in der Spule.
\begin{equation}
	\Delta I_{L} = \frac{V_{OUT}}{L \cdot f_{SW}} \cdot (1-\frac{V_{OUT}}{V_{IN(MAX)}})
	\label{eqn:ripple_current}
\end{equation}
Zudem sind weitere Formeln gegeben, welche die Auswirkungen des Rippelstroms beschreiben.
\begin{equation}
	I_{L(PEAK)} = I_{LOAD(MAX)} + \frac{1}{2}\Delta I_{L}
\end{equation}
\begin{equation}
	I_{OUT(MAX)} = I_{LIM} - \frac{\Delta I_{L}}{2}
\end{equation}
Diese sagen also zwei Dinge aus:
\begin{itemize}
	\item Je höher der Rippelstrom, desto höher der maximale Spulenstrom.
	\item Je höher der Rippelstrom, desto kleiner der maximale Ausgangsstrom.
\end{itemize}
Daher ist es ein kleiner Rippelstrom zu bevorzugen. Gemäss Formel \ref{eqn:ripple_current} kann dies entweder durch eine höhere Schaltfrequenz, eine grössere Spule oder durch eine kleinere Differenz zwischen Vout und Vin erreicht werden. Es wurde daher eine 10uH-Spule, welche somit fast drei mal so hoch ist wie der Wert aus der Faustregel. Mit allen Werten in Formel \ref{eqn:ripple_current} eingesetzt ergibt das $\Delta I_{L} = \frac{15V}{10uH \cdot 2MHz} \cdot (1-\frac{15V}{24V}) = \textbf{281mA}$.\\Letztenendes wurde die Spule \href{https://www.we-online.com/components/products/datasheet/7443251000.pdf}{7443251000} (7.2A rated current) für die 15V-Speisung und \href{https://www.we-online.com/components/products/datasheet/74404043010A.pdf}{74404043010A} für die 5V-Speisung, beide von Würth Elektronik, ausgewählt.
\subsubsection{Kaskadierung von OUT1 und OUT2}
Generell ist es bei Audioanwendungen angebracht, die Speisung der Endstufe als letztes einzuschalten, da dadurch Pop-Geräusche bei Signalbeginn tendenziell verhindert werden. Bei der Simulation wurde zudem beobachtet, dass der Spitzenstrom der Speisung verringert werden kann wenn die Speisungen nacheinander eingeschaltet werden. Dazu musste lediglich das \textit{PG1}-Signal mit dem \textit{EN2}-Eingang verbunden werden.
\subsubsection{Verbessertes Transientenverhalten}
Für das in Abbildung \ref{pic:transient_behv_LT8650S} zu sehende Transientenverhalten wurde der \textit{VC2}-Pin über eine RC-Serieschaltung auf GND verbunden. Die Werte wurden aus dem Datenblatt übernommen\footnote{Hier könnte noch weiter optimiert werden, jedoch war dies aus Zeitgründen nicht mehr möglich.}.
\subsubsection{Bessere Stabilität durch Anpassung des FB-Kondensators}
Es wurde zudem beobachtet, dass die Schwinganfälligkeit erheblich verbessert werden konnte, wenn für C7 ein 220pF-Kondensator eingesetzt wurde.
\subsubsection{Eingangsnetzwerk}
Im Datenblatt wird am Ende des Kapitels \textit{Input Capacitor} eine Überspannung erwähnt, die durch eine Kombination von Keramikkondensatoren mit niedrigem ESR und einer parasitären Kabelinduktion entsteht:
\begin{quote}
	[...] A second precaution regarding the ceramic input capacitor concerns the maximum input voltage rating of the LT8650S. A ceramic input capacitor combined with trace or cable inductance forms a high quality (under damped) tank circuit. If the LT8650S circuit is plugged into a live supply, the input voltage can ring to twice its nominal value, possibly exceeding the LT8650S’s voltage rating. This situation is easily avoided (see Analog Devices Application Note 88).
	- \cite{LT8650S_Datasheet}
\end{quote}
Die erwähnte Überspannung entsteht genau dann, wenn ein Netzteil über ein langes Kabel und einem Schalter mit noch ungeladenen Keramikkondensatoren verbunden ist. Genau das ist ja die Situation hier! In der Application Note\footnote{Siehe: \cite{OvervoltageWithCerCaps}} wird die Anwendung eines Snubber-Netzwerkes, also eines grösseren Kondensators mit einem kleinen (0.5Ohm) Widerstand in Serie. Dieser sozusagen künstlich verschlechterte Kondensator sorgt dafür, dass Energie weiterhin aus dem Netzwerk entzogen sobald die Keramikkondensatoren geladen sind. Ansonsten würde die Spule einen stark abfallenden Strom \textquotedbl{}sehen\textquotedbl{} und eine Spannung erzeugen, welche sich zu den bereits vorhandenen 24V hinzuaddiert. In Abbildung \ref{pic:startup_comparision} ist der direkte Vergleich zwischen dem Aufstartverhalten mit und ohne Snubber zu sehen. Dabei wurde ein 47uH-Kondensator und einem 0.5Ohm-Widerstand in Serie ausgewählt. Zu beachten ist, dass bei \ref{pic:startup_nosnubber} nicht nur eine Spannungsüberhöhung stattfindet sondern auch Strom zurück in die Speisung fliesst! Das in \ref{pic:startup_withsnubber} zu sehende Aufstartverhalten wird allerdings erkauft mit einem erhöhtem Eingangsstrom. Da dieser Strom allerdings nur wenige $\mu$s andauert, kann dieser als unproblematisch erachtet werden bzw. sollten Netzteile mit einem Einschaltstrom umgehen können.\\
\begin{figure}[H]
	\centering
	\vspace{-6mm}
	\includegraphics[width=\textwidth*3/4]{pictures/simulation_input_network.png}
	\caption{Das Eingangsnetzwerk}
	\label{pic:simulation_input_network}
\end{figure}
\begin{figure}[H]
	\centering
	\begin{subfigure}{\textwidth*6/13}
		\includegraphics[trim={0 5mm 0 0},clip,width=\textwidth]{pictures/startup_Schaltregler_noSnubber_L2_10uH.pdf}
		\caption{Aufstartverhalten ohne Snubber}
		\label{pic:startup_nosnubber}
	\end{subfigure}
	\begin{subfigure}{\textwidth*6/13}
		\includegraphics[trim={0 5mm 0 0},clip,width=\textwidth]{pictures/startup_Schaltregler_withSnubber_L2_10uH.pdf}
		\caption{Aufstartverhalten mit Snubber}
		\label{pic:startup_withsnubber}
	\end{subfigure}
	\caption{Der Snubber stabilisiert das System}
	\label{pic:startup_comparision}
\end{figure}
\noindent Zusätzlich wurden zwei 4.7uH Keramikkondensatoren gemäss dem Datenblatt\footnote{Siehe Input Capacitor aus \cite{LT8650S_Datasheet}} verwendet.
\subsubsection{Efficiency Report}
LTSpice bietet auch die Möglichkeit, einen Efficiency Report zu generieren. Bei diesem wird für jedes Bauteil sowohl RMS- und Peak-Ströme als auch die Verlustleistung berechnet. Dabei werden bei reaktiven Bauteilen ein gesondertes Verfahren angewandt. In Abbilung \ref{pic:efficiency_report} ist der gesamte Report zu sehen.
\begin{figure}[H]
	\vspace{-5mm}
	\centering
	\includegraphics[trim = {2cm 11cm 10cm 2cm}, clip, width=\textwidth*2/4]{pictures/efficiency_report.pdf}
	\caption{Der Efficiency Report der Simulation}
	\label{pic:efficiency_report}
\end{figure}
Zu beachten ist, dass dieser von einer einzelnen Last ausgeht. Da hier mehrere Lasten anliegen, kann der Report nicht die \textquotedbl{}richtige\textquotedbl{} Last bestimmen. Die Effizienz kann jedoch einfach rechnerisch bestimmt werden:\\
$\mu_{Schaltregler} = \frac{P_{out}}{P_{in}} = \dfrac{\sum_{n=1}^{N_{Loads}}P_{Load}}{P_{in}}\rightarrow\mu_{Schaltregler} = \frac{5.6W + 3\cdot 10.849W}{39W} = 97.81\%$
\subsubsection{Kontrolle Wärmeverhalten}
Mit den Daten aus dem Efficiency Report konnte nun abgeschätzt werden, wie warm der Schaltregler während des Betriebs werden könnte und ob dieser u.U. einen Kühlkörper brauchte. Da eine genaue Simulation zeitaufwändig und keine Software verfügbar war, wurde eine andere Methode angewandt:\\
Auf der Webseite \cite{pcbtemp_calculator} war nebst anderen ein Berechnungstool verfügbar, mit welchem die voraussichtliche Junction-Temperatur\footnote{Also die Temperatur zwischen zwei Materialien im Silizium-Chip. Überschreiten der maximalen Junction-Temperatur führt zur Zerstörung des Bauteils.} eines Chips berechnet werden kann. Diese ist von sehr vielen Umgebungsgrössen wie grösse des PCBs, Anteil der Kupferfläche, Anzahl Vias, Luftstrom etc. abhängig und recht aufwändig zu berechnen. Mit diesem Tool konnten nun die Dimensionen sowie Kenndaten eingetragen und alle Berechnungen\footnote{Die genauen Berechnungsmethoden waren hier hinterlegt: \href{https://www.heatsinkcalculator.com/blog/how-to-calculate-the-thermal-resistance-of-a-pcb/}{How to calculate the thermal resistance of a PCB}} automatisch durchgeführt werden. Somit konnte nicht nur die Junction-Temperatur abgeschätzt werden, sondern auch der thermische Widerstand zwischen Junction und Umgebung, also welche Temperaturdifferenz pro Watt Verlustleistung entsteht.\\Abbildung \ref{pic:temp_calc} zeigt alle eingegebenen Daten, wovon einige zu diesem Zeitpunkt Schätzungen waren. Das Resultat (\ref{pic:temp_calc_result}) zeigte bei 0.5W Verlustleistung eine Junctiontemperatur von 36.7 °C. Im Datenblatt des LT8650S wird eine maximale Junctiontemperatur von 125° bis 150° C angegeben\footnote{Es gab verschiedene Modellnummern, welche zum Teil einen erweiterten Temperaturbereich besitzen.}.
\begin{center}
	\begin{minipage}{\textwidth*7/8}
		\centering
		\vspace{6mm}
		{\large Somit wurde entschieden, dass der Schaltregler \textbf{kein Kühlkörper} braucht.}
		\vspace{6mm}
	\end{minipage}
\end{center}
\begin{figure}[H]
	\centering
	\begin{subfigure}{\textwidth*11/13}
		\includegraphics[width=\textwidth]{pictures/temp_calc_LT8650S_PCBdata.png}
		\caption{Daten des PCBs}
		\label{pic:temp_calc_pcbdata}
		\vspace{3mm}
	\end{subfigure}\\
	\begin{subfigure}{\textwidth*11/13}
		\includegraphics[width=\textwidth]{pictures/temp_calc_LT8650S_Airdata.png}
		\caption{Daten der Umgebungsluft}
		\label{pic:temp_calc_airdata}
		\vspace{3mm}
	\end{subfigure}
\end{figure}
\begin{figure}[H]\ContinuedFloat
\centering
	\begin{subfigure}{\textwidth*11/13}
		\includegraphics[width=\textwidth]{pictures/temp_calc_LT8650S_Devicedata.png}
		\caption{Daten des Chips}
		\label{pic:temp_calc_devicedata}
		\vspace{3mm}
	\end{subfigure}\\
	\begin{subfigure}{\textwidth*11/13}
		\includegraphics[width=\textwidth]{pictures/temp_calc_LT8650S_Viadata.png}
		\caption{Daten der Vias}
		\label{pic:temp_calc_viadata}
		\vspace{3mm}
	\end{subfigure}\\
	\begin{subfigure}{\textwidth*11/13}
		\includegraphics[width=\textwidth]{pictures/temp_calc_LT8650S_Heatsinkdata.png}
		\caption{Daten des Kühlkörpers (nicht vorhanden)}
		\label{pic:temp_calc_heatsindkdata}
		\vspace{3mm}
	\end{subfigure}\\
	\begin{subfigure}{\textwidth*11/13}
		\includegraphics[width=\textwidth]{pictures/temp_calc_LT8650S_Result.png}
		\caption{Resultat der Temperaturberechnung}
		\label{pic:temp_calc_result}
		\vspace{3mm}
	\end{subfigure}
	\caption{Berechnungsdaten für die Temperaturberechnung}
	\label{pic:temp_calc}
\end{figure}