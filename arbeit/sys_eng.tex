\subsection{Terminplanung}
Für das Projekt wurden nun abgegrenzte Arbeitspakete definiert und diese in einen Zeitplan überführt. Dabei wurde darauf geachtet, dass wichtigere bzw. kritische Pakete (z.B. die Variantenauswahl) mehr Zeit bekamen. Als Hilfmittel wurde zudem die Projektfunktion von github.com verwendet. Dieses Tool bietet nicht nur den Vorteil einer grafischen Darstellung (Roadmap, Burn-up etc.), sondern auch dass jedes Paket mit einer Historie, Kommentaren (auch von dritten), Files, Links sowie Referenzen untereinander ergänzt werden. So kann der Projektverlauf dynamisch auf jedes einzelnes Paket hin verfolgt werden.\\In Tabelle \ref{terminplan} sind alle Arbeitspakte, deren Zeitrahmen sowie den jeweiligen Github-Links nochmals tabellarisch dargestellt.
Für das Projekt wurden nun abgegrenzte Arbeitspakete definiert und diese in einen Zeitplan überführt. Dabei wurde darauf geachtet, dass wichtigere bzw. kritische Pakete (z.B. die Variantenauswahl) mehr Zeit bekamen. Als Hilfmittel wurde zudem die Projektfunktion von github.com verwendet. Dieses Tool bietet nicht nur den Vorteil einer grafischen Darstellung (Roadmap, Burn-up etc.), sondern auch dass jedes Paket mit einer Historie, Kommentaren (auch von dritten), Files, Links sowie Referenzen untereinander ergänzt werden. So kann der Projektverlauf dynamisch auf jedes einzelnes Paket hin verfolgt werden.\\In Tabelle \ref{terminplan} sind alle Arbeitspakte, deren Zeitrahmen sowie den jeweiligen Github-Links nochmals tabellarisch dargestellt.
\begin{table}[H]
	\centering
	\caption{Terminplanung in tabellarischer Form}
	\setstretch{1.5}
	\begin{tabularx}{412pt}{|l|l|l|l|}
		\textbf{Arbeitspaket} & \textbf{URL} & \textbf{Startdatum} & \textbf{Enddatum} \\\hline
		\textbf{Terminplanung} & \href{https://github.com/Violabitch5/Nathophone\_doc/issues/1}{Link} & Sep 4, 2025 & Sep 5, 2025 \\ 
		\textbf{IST-Zustandsanalyse} & \href{https://github.com/Violabitch5/Nathophone\_doc/issues/4}{Link} & Sep 6, 2025 & Sep 7, 2025 \\ 
		\textbf{Zieldefinition} & \href{https://github.com/Violabitch5/Nathophone\_doc/issues/3}{Link} & Sep 8, 2025 & Sep 9, 2025 \\ 
		\textbf{Zielgewichtung} & \href{https://github.com/Violabitch5/Nathophone\_doc/issues/5}{Link} & Sep 10, 2025 & Sep 11, 2025 \\ 
		\textbf{Varianten- \& Risikoanalyse} & \href{https://github.com/Violabitch5/Nathophone\_doc/issues/6}{Link} & Sep 12, 2025 & Sep 16, 2025 \\ 
		\textbf{Variantenauswahl} & \href{https://github.com/Violabitch5/Nathophone\_doc/issues/7}{Link} & Sep 17, 2025 & Sep 17, 2025 \\ 
		\hdashline
		\textbf{Kennzahlberechnung / Limits} & \href{https://github.com/Violabitch5/Nathophone\_elec/issues/1}{Link} & Sep 18, 2025 & Sep 21, 2025 \\ 
		\textbf{Bauteileevaluation} & \href{https://github.com/Violabitch5/Nathophone\_elec/issues/2}{Link} & Sep 22, 2025 & Sep 30, 2025 \\ 
		\textbf{Bauteilauswahl} & \href{https://github.com/Violabitch5/Nathophone\_elec/issues/3}{Link} & Oct 1, 2025 & Oct 4, 2025 \\ 
		\textbf{Print Schema draft} & \href{https://github.com/Violabitch5/Nathophone\_elec/issues/4}{Link} & Oct 5, 2025 & Oct 11, 2025 \\ 
		\textbf{Print Schema v1} & \href{https://github.com/Violabitch5/Nathophone\_elec/issues/5}{Link} & Oct 12, 2025 & Oct 18, 2025 \\ 
		\textbf{Print Layout v1} & \href{https://github.com/Violabitch5/Nathophone\_elec/issues/6}{Link} & Oct 19, 2025 & Oct 26, 2025 \\ 
		\textbf{Peer-review Schema und Layout} & \href{https://github.com/Violabitch5/Nathophone\_elec/issues/7}{Link} & Oct 27, 2025 & Oct 31, 2025 \\ 
		\textbf{Print Schema Final} & \href{https://github.com/Violabitch5/Nathophone\_elec/issues/8}{Link} & Nov 1, 2025 & Nov 3, 2025 \\ 
		\textbf{Print Layout final} & \href{https://github.com/Violabitch5/Nathophone\_elec/issues/9}{Link} & Nov 3, 2025 & Nov 6, 2025 \\ 
		\textbf{Gerber-Daten generieren und Printbestellung} & \href{https://github.com/Violabitch5/Nathophone\_elec/issues/10}{Link} & Nov 7, 2025 & Nov 7, 2025 \\ 
		\textbf{Konstruktion des Korpus fertigstellen} & \href{https://github.com/Violabitch5/Nathophone\_mech/issues/1}{Link} & Nov 8, 2025 & Nov 10, 2025 \\ 
		\textbf{Herstellung Korpus-Teile} & \href{https://github.com/Violabitch5/Nathophone\_mech/issues/2}{Link} & Nov 11, 2025 & Nov 15, 2025 \\ 
		\textbf{Zusammenbau Korpus} & \href{https://github.com/Violabitch5/Nathophone\_mech/issues/3}{Link} & Nov 16, 2025 & Nov 22, 2025 \\ 
		\textbf{Bestückung + Lötarbeit Print} & \href{https://github.com/Violabitch5/Nathophone\_elec/issues/11}{Link} & Nov 23, 2025 & Nov 30, 2025 \\ 
		\textbf{Funktionstest Print} & \href{https://github.com/Violabitch5/Nathophone\_elec/issues/12}{Link} & Dec 1, 2025 & Dec 4, 2025 \\ 
		\textbf{Zusammenbau Komplettsystem \& Programmierung} & \href{https://github.com/Violabitch5/Nathophone\_mech/issues/4}{Link} & Dec 4, 2025 & Dec 18, 2025 \\ 
		\textbf{Funktionstests Komplettsystem} & \href{https://github.com/Violabitch5/Nathophone\_elec/issues/13}{Link} & Dec 18, 2025 & Dec 25, 2025 \\ 
		\textbf{Schlussmessung \& Auswertungen} & \href{https://github.com/Violabitch5/Nathophone\_elec/issues/14}{Link} & Dec 25, 2025 & Dec 31, 2025 \\ 
		\hdashline
		\textbf{Audruck (Plakat und Arbeit)} & \href{https://github.com/Violabitch5/Nathophone\_doc/issues/8}{Link} & Jan 1, 2026 & Jan 6, 2026 \\ 
		\textbf{Abgabe Diplomarbeit} & \href{https://github.com/Violabitch5/Nathophone\_doc/issues/2}{Link} & Jan 7, 2026 & Jan 7, 2026
	\end{tabularx}
	\label{terminplan}
\end{table}