\subsection{Abstract}
\subsection{Theorie}
\subsubsection{Das Prinzip des Linienstrahlers}
Ein jeder kennt die markanten Lautsprechersysteme von grösseren Eventveranstalltungen. Jedoch kennen nur die wenigsten deren Wirkungsweise, da immer mehr die visuellen Effekte im Vordergrund stehen. Jedoch könnte ein Stadion ohne diese Technologie wohl kaum effizient und in genügender Audioqualität beschallt werden.\\Die Wirkungsweise eines solchen \textit{Line Arrays} ist schnell erklärt: Mittel- und Hochtontreiber werden gleichmässig auf einer Linie angeordnet so dass sich durch Schallinterferenzen die einzelnen Wellensignale gezielt gegenseitig auslöschen und dadurch akustische Energie nur in bestimmte Richtungen abgestrahlt wird. \\Wie so oft ist dieser Effekt allerdings von mehreren Faktoren abhängig: Zum einen verschieben sich mit einer Änderung der Frequenz alle Phasenlagen, so dass sich alle Auslöschungszonen verschieben. Zum anderen spielen die genauen Dimensionen, Charakteristiken und Abstände zwischen den einzelnen Klangquelle eine entscheidende Rolle. So wirkt ein Line Array nur in einem bestimmen Frequenzband als Linienstrahler. Unterhalb dieses Frequenzbereichs interferieren die einzelnen Schallwellen wegen der langen Wellenlängen kaum noch und agieren mit tieferer Frequenz zunehmend als eine einzelne sphärische Schallquelle. Oberhalb des Frequenzbereiches ist die Wellenlänge kurz im Vergleich zu den Dimensionen der einzelnen Arrayelemente. Dadurch kommt es bereits zu Auslöschungen noch im Nahfeld des Elements, wodurch die Schallenergie direkt senkrecht abgestrahlt wird und nicht mehr mit dem restlichen Array interagiert.
{\color{red} ABBILDUNG SIMULATION}
\paragraph{Dynamisches Line Array}
Dieser  Durch geschickte Verzögerungen der elektrischen Signale lässt sich die Abstrahlcharakteristik auf modernen Linenstrahlern auch softwaremässig programmieren\footnote{\color{red}FOHNLINK}.
\paragraph{Dreidimensionales Beam Steering}
\subsubsection{Das Prinzip des Monochord}
\subsubsection{Signaltransport}
\subsubsection{Elektronische Klangerzeugung}
