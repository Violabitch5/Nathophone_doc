\subsection{Abstract}
\subsection{Theorie}
\subsubsection{Das Prinzip des Linienstrahlers}
Ein jeder kennt die markanten Lautsprechersysteme von grösseren Eventveranstalltungen. Jedoch kennen nur die wenigsten deren Wirkungsweise, da oft die visuellen Effekte im Vordergrund stehen. Jedoch könnte ein Stadion ohne diese Technologie wohl kaum effizient und in genügender Audioqualität beschallt werden.\\Die Wirkungsweise eines solchen \textit{Line Arrays} ist schnell erklärt: Lautsprecherelemente werden gleichmässig auf einer Linie angeordnet so dass sich die Schallwellen gezielt gegenseitig auslöschen und dadurch akustische Energie nur in eine bestimmte Richtung abgestrahlt wird. Das ganze kann man sich sozusagen als \textquotedbl{}akustischen Scheinwerfer\textquotedbl{} vorstellen.\\Wie so oft ist dieser Effekt allerdings von mehreren Faktoren abhängig und nur innerhalb eines bestimmten Frequenzbandes wirksam. Dieses wird im wesentlichen durch die Arraylänge und den Abstand zwischen den Arrayelementen bestimmt. Abbildung \ref{pics:RH_linearray_simu} zeigt die Simulation des Herstellers
\href{https://renkus-heinz.com/}{Renkus-Heinz}.
\begin{figure}[H]
	\centering
	\begin{subfigure}{\textwidth*6/13}
		\centering
		\includegraphics[width=\textwidth]{pictures/RH_whitepaper_simulation2.png}
		\caption{Verhalten bei einer \textbf{Arraylänge} von $\frac{\lambda}{2}$}
	\end{subfigure}
	\vspace{2mm}
	\begin{subfigure}{\textwidth*6/13}
		\centering
		\includegraphics[width=\textwidth]{pictures/RH_whitepaper_simulation.png}
		\caption{Verhalten bei einer \textbf{Arraylänge} von $2\lambda$}
		\vspace{2mm}
	\end{subfigure}\\
	\begin{subfigure}{\textwidth*6/13}
		\centering
		\includegraphics[width=\textwidth]{pictures/RH_whitepaper_simulation3.png}
		\caption{Verhalten bei einem \textbf{Elementabstand} von $\frac{\lambda}{2}$}
	\end{subfigure}
	\begin{subfigure}{\textwidth*6/13}
		\centering
		\includegraphics[width=\textwidth]{pictures/RH_whitepaper_simulation4.png}
		\caption{Verhalten bei einem \textbf{Elementabstand} von $2\lambda$}
	\end{subfigure}
	\caption{Simulation des Herstellers Renkus-Heinz. Siehe: \cite{R-H_LineArrayTheory}}
	\label{pics:RH_linearray_simu}
\end{figure}
\paragraph{Beam Steering}
Wenn alle Elemente des Line Arrays das genau gleiche Signal, mit der gleichen Phasenlage erhalten entsteht das in Abbildung \ref{pics:RH_linearray_simu} zu sehende Verhalten, also im Idealfall eine Ausrichtung von 90°. Wenn aber nun jedes Element ein eigenes Signal erhält, kann durch geschickte Verzögerungen und Filterungen die Ausrichtung verändert werden. Abbildung \ref{pics:MeyerSoung_BeamSteering} zeigt eine Simulation des Herstellers \href{https://meyersound.com/}{Meyer Sound}. Da diese Ausrichtung sich mit der Frequenz ändert (In der Abbildung zu sehen), müssen die Verzögerungen für jede Frequenz neu berechnet werden.
\begin{figure}[H]
	\centering
	\includegraphics[width=\textwidth-4cm]{pictures/meyersound_beemsteering_simulation.png}
	\caption{Beam Steering Simulation. Quelle: \cite{MeyerSound_DSP_Beam_Steering}}
	\label{pics:MeyerSoung_BeamSteering}
\end{figure}
\paragraph{Dreidimensionales Beam Steering} Das beschriebene Prinzip beruht auf einer Linienförmigen Anordnung der Elemente und ermöglicht eine steuerbare Ausrichtung auf einer Achse. Es können natürlich ohne weiteres auch mehrere Line Arrays nebeneinander aufgebaut werden, wodurch eine zweidimensionale Anordnung entsteht und Schall in jede beliebige Richtung ausgesendet werden kann. Dies ist allerdings nicht Bestandteil dieser Arbeit.
\subsubsection{Das Prinzip des Monochord}
\subsubsection{Signaltransport}
\subsubsection{Elektronische Klangerzeugung}
