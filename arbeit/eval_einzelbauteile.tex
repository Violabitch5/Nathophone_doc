\subsection{Evaluation Einzelkomponenten}
Hier sind die Begründungen aufgeführt, wie einzelne Teile oder Komponenten ausgewählt wurden. Dabei spielte zu einem grossen Teil auch die Verfügbarkeit eine wichtige Rolle.
\subsubsection{ext. Netzteil}
Hier musste ein genügend Leistungsstarkes Netzteil gewählt werden. Gemäss Abbildung \ref{pic:Leistungsbudget} sind dies zwischen 72 und 120W. Die Wahl viel auf  \href{https://www.bicker.de/files/downloads/datenblatt/bet-0900-t_e.pdf}{BET-0900}.
\subsubsection{Speisungsstecker}
Der Stecker musste von der Bauform her auf das Netzteil passen, genügend Spannungsfestigkeit haben (min. 36V) und den erwartbaren Spitzenstrom aushalten. Da der Spitzenstrom erst noch simuliert werden musste, viel die Wahl zunächst auf den \href{https://www.cuidevices.com/product/resource/pj-063ah.pdf}{PJ-063AH}.
\subsubsection{Speisungsschalter}
Es wurde entschieden, dass die gesamte Speisung über einen Hauptschalter ein- oder ausgeschaltet werden soll. Es wäre auch möglich gewesen, die EN-Pins des Schaltregler dafür zu nutzen. Jedoch wurde bevorzugt, dass sich im AUS-Zustand auf gar keinen Fall Strom in die Schaltung fliessen kann. Die wahl viel zunächst auf den \href{https://www.nkkswitches.com/pdf/SW-1.pdf}{SW3001A} welcher allerdings nicht als Print-Montage erhältlich war und somit über Litzen verbunden werden musste.
\subsubsection{Überspannungsschutz}
Um möglicherweise auftretende Transienten (z.B. bei Einschaltvorgängen) abzublocken, wurde eine TVS-Supressor eingesetzt. Dabei musste eine Klemmspannung gewählt werden, die höchstens der kleinsten maximalen Eingangsspannung eines Bauteils entsprach. Die Wahl viel auf den \href{https://www.littelfuse.com/assetdocs/tvs-diode-tpsmb-datasheet?assetguid=E7A196A8-A225-4724-AB76-7C22E0D48557}{TPSMB2616CA} mit einer Klemmspannung von 39.5V.
\subsubsection{Speisungsfilter}
Um die erwartbare Rippelspannung des Netzteils zu glätten und kurze Spitzen des eigenen Schaltreglers abzufangen wurde ein Filterkondensator vorgesehen. Dabei wurde ein 10nF Filmkondensator gewählt, da Metallfilm-kondensatoren keine DC-Degradierung erfahren und zudem selbstheilende Effekte\footnote{Siehe: \cite{healingofhilmcaps}} haben. Die Wahl fiel auf \href{https://industrial.panasonic.com/ww/products/pt/film-cap-electroequip/models/ECHU1H103JX5}{ECH-U1H103JX5}, welcher eine Spannungsfestigkeit von 50 VDC hat.
\subsubsection{Verpolungsschutz}
Ein effektiver Verpolungsschutz kann mit einem N-Kanal Mosfet realisiert werden. Dieser muss eine genügend grosse Drain-Source und Gate-Source Spannungsfestigkeit aufweisen und den erwartbaren Spitzenstrom aushalten. Die Wahl viel auf den \href{https://www.onsemi.com/pdf/datasheet/fqu11p06-d.pdf}{FQD11P06TM} mit einem $ID_{max}$ von 9.4A und $ID_{pulsed}$ von 37.6A.
\subsubsection{interne 3.3V-Speisung \& Supply Switchover}
Die gesamte Signalanbindung ging bislang von einem Milan-Modul aus, welches die 3.3V-Spannung erzeugt und zurück auf das Driverboard speist. Was aber, wenn dieses Modul in einer künftigen Anwendung nicht vorhanden ist? Dann wäre plötzlich keine 3.3V-Speisung vorhanden. Daher wurde entschieden, eine interne 3.3V-Speisung mit einem LDO-Regler einzubauen. Zudem soll eine einfache Schaltung zur automatischen Umschaltung der Speisung, mit Priorität auf der externen Speisung, eingebaut werden. Somit ist in jedem Fall eine 3.3V-Spannung, welche die Endstufen für deren digitalen Teil benötigen, gegeben. Als LDO wurde der \href{https://mm.digikey.com/Volume0/opasdata/d220001/medias/docus/6642/1016_SPX3819.pdf}{SPX3819M5-L-3-3/TR} 0.5A nom. und Überstrom ausgewählt. Als für die Speisungsumschaltung wurden zwei P-Kanal MOS-FETs (FDN340P) und ein N-Kanal MOS-FET (N7002N) verwendet.
\subsubsection{Group SDZ}
Die Endstufen besitzen einen Low-Aktiven Shutdown-Eingang. Da es bei einem Line-Array bei einem Fehlerfall auf einem Kanal keinen Sinn macht, wenn die restlichen Endstufen weiterlaufen wurden alle Shutdown-Signale miteinander verbunden. Daher wurde das Signal \textit{Group\_SDZ} benannt. Das Milan-Modul besitzt ebenfalls einen ADC-Reset Ausgang (\textit{ADC\_RST\_N\_OUT}), dessen Verhalten vom Hersteller wie folgt beschrieben wird:
\begin{quote}
	The reset line allows to reset peripherals on the extension cards. The reference firmware pulls the reset line low on initialisation or whenever a sample rate change occurs.\\- \cite{MT32_EVK_Datasheet}
\end{quote}
Aus Erfahrungswerten wurde mittels einer Lötbrücke die Möglichkeit eingebaut, dieses Signal mit dem \textit{Group\_SDZ} Signal zu verbinden oder zu trennen.
\subsubsection{Power-Good LED}
Die beiden Power Good-Signale des Schalreglers wurden zusammengeschlossen und verwendet, um eine grüne LED-Anzusteuern. Diese signalisiert somit den Zustand der Speisespannung.
\subsubsection{Fault LEDs}
Im Gegensatz zum Group-SDZ wurde jeder \textit{FAULTZ}-Ausgang der Endstufe auf eine separate rote LED verbunden. Somit ist bei Fehlerfällen ersichtlich, welche Endstufe eine Störung ausgelöst hat.
\subsubsection{Output Connectors}
Für die Verbindung zu den Excitern wurden Leiterplattenklemmen von Phoenix Contact verwendet: \href{https://www.phoenixcontact.com/us/products/1862042/pdf}{1862042}. Diese sind nur 8mm hoch, haben einen Abgangswinkel von 45° und sind mit 8A und 300V genügend robust spezifiziert. Zudem können die Kontakte ohne Werkzeug verbunden werden.
\subsubsection{GPIO LED}
Das Milan-Modul besitzt zudem einen GPIO-Ausgang, welcher vom Hersteller mit \textquotedbl{}Reserved for future use.\textquotedbl{} beschrieben wird. Da die Funktionalität u.U. auch konfigurierbar sein könnte, wurde trotzdem eine gelbe LED vorgesehen. So könnte z.B. der Verbindungsstatus angezeigt werden.
\subsubsection{opt. ext. TDM Input}
Mittels Stifleisten soll auch die Möglichkeit bestehen, aus einer anderen Quelle ein TDM-Signal einspeisen zu können. Wie sehr diese Stiftleiste HF-Mässig eine Störung hervorruft oder gar aussendet muss noch eruiert werden.
\subsubsection{Testsignale}
Zur Fehlersuche oder auch externe Module wurden einige weitere Stiftleisten vorgesehen. Mittels rücksetzbaren PTC-Sicherungen können auch die Speisungen auf die Stiftleisten geführt werden und verringern somit die Gefahr einer unabsichtlichen Zerstörung.
\subsubsection{opt.: System Monitor I2C-ADC}
Es kam auch die Idee auf, mittels eines einfachen I2C-ADCs alle Speisungs- und Signalpegel digital überwachen zu können. Es war allerdings fraglich, ob dieser Nutzen den Aufwand für die korrekte Umsetzung rechtfertigt.