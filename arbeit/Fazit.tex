Diese Arbeit bot einige Herausforderungen, einiges an Umfang und forderte Entscheidungen in mehreren Fachgebieten. Trotz der Tatsache, dass es schlussendlich aus Zeitgründen nicht mehr ausreichte, das MILAN-Modul komplett einzubinden und zu dokumentieren, war das Projekt in einem sehr vielversprechenden Zustand. Leider musste das namensgebende Konzept der Arbeit bereits sehr früh verlassen werden, da die Konstruktion mit einer Saite zu viele Risiken und zu viel Zeitaufwand verursacht hätte. Nichtsdestotrotz konnten die in \ref{sec:zielgewichtung} beschriebenen Primärziele weitgehend umgesetzt oder zumindest Massnahmen dazu ergriffen werden. Dahingehend war es durchaus möglich, das System im Nachgang der Arbeit noch fertigzustellen und zu verifizieren.\\Gerade im Bereich des Beam-Forming kann hier noch sehr viel verfeinert und abgestimmt werden. Zudem hat das Projekt auch Potential zur Erweiterung, z.B. könnte die gesamte Signalverarbeitung direkt auf dem XMOS-Chip implementiert werden oder es könnten noch Leistungsstärkere Exciter verwendet werden.\\Somit bietet der aktuelle Stand der Dinge einen sehr guten Rahmen für die weitere Zukunft.