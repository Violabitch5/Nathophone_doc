Die Variantenanalysen zeigt klar, dass \textbf{Variante D: Einfache Anwendung} aus Risikogründen, Nutzwertanalyse und der Kosten-Nutzen die vielversprechendste Variante ist. Die Kombination aus fixfertigen Modulen, wenigen Neuentwicklungen und wenigen Komponenten führt in vielen Belangen zu vorteilhaften Eigenschaften. Ein weiterer Vorteil ist, dass durch den Einsatz von Excitern sich die Konstruktion erheblich vereinfacht, da keine Saite gespannt oder fixiert werden muss und keine sich bewegende Teile von aussen zugänglich sind.
\paragraph{Anpassungen}Nichtsdestotrotz kann diese Variante noch weiter optimiert und kombiniert werden. So können zum Beispiel mehrere Materialien für das Gehäuse verwendet werden. Daher wurde diese Variante auf die Primärziele hin noch weiter angepasst. Nachfolgend sind daher die Primärziele mit den entsprechenden Massnahmen aufgelistet:\\
\begin{figure}[H]
	\centering
	\setstretch{1.5}
	\begin{tabularx}{\textwidth*4/5}{lX}
		{\large \textbf{Ziel}} & {\large Massnahme}\\
		\midrule\vspace{-6mm}\\
		\textbf{G: Reduziertes Brandrisiko}&Montage der stromführenden Bauteile auf PMMA\\
		\vspace{2mm}\textbf{D.2: Mobilität}&Einsatz eines externen USB-MILAN Dongles für die Verbindung zwischen PC und Gerät. Somit ist kein Switch vonnöten.\\
		\vspace{2mm}\textbf{A: Direktionale Abstrahlung}&Externes Netzteil für grösseres Leistungsbudget\\
		\bottomrule
	\end{tabularx}
	\caption{Weitere Massnahmen}
\end{figure}

