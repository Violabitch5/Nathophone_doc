\subsection{Rendering PCB}
Mittels den CAD Programmen KiCAD und Fusion360 konnte das Driverboard separat und im Gehäuse gerendert werden. Abbildung \ref{pic:render_pcb} zeigt das fertige Driverboard, Abbildung \ref{pic:pcb_incase_1} das Gehäuse mit den beiden PCBs von vorne und Abbildung \ref{pic:pcb_incase_2} als Detailansicht\footnote{Der Schalter passte aus reinem Zufall genau in die Aussparung.}.
\begin{figure}[H]
	\centering
	\includegraphics[ width=\textwidth]{pictures/Nathophone_DriverBoard_PCB.png}
	\caption{Render des Driverboards}
	\label{pic:render_pcb}
\end{figure}
\begin{figure}[H]
	\centering
	\begin{subfigure}{\textwidth}
		\centering
		\includegraphics[width=\textwidth]{pictures/Nathophone_withPCBs_1.png}
		\caption{Ansicht von vorne}
		\label{pic:pcb_incase_1}
		\vspace{3mm}
	\end{subfigure}\\
	\begin{subfigure}{\textwidth}
		\centering
		\includegraphics[width=\textwidth]{pictures/Nathophone_withPCBs_2.PNG}
		\caption{Detailansicht}
		\label{pic:pcb_incase_2}
		\vspace{3mm}
	\end{subfigure}\\
	\caption{Renderings mit Gehäuse}
\end{figure}