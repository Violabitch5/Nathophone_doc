Es wurde eine Methode gesucht, das Verhalten eines Line Arrays zumindest ansatzweise zu visualisieren. Insbesondere, ob und welche Signalbearbeitung nötig ist. Dabei wurden im Internet teils sehr komplexe physikalische Berechnungsmethoden gefunden. Die einfachste Berechnungsmethode wurde allerdings in einer Masterarbeit eines kolumbianischen Studenten entdeckt. Abbildung \ref{pic:Diagramm_Salazar} ist dabei die Grundlage für Gleichung \ref{equ_pressure} und entspricht der Gleichung \textbf{36} in der Arbeit.
\begin{figure}[H]
	\centering
	\includegraphics[width=\textwidth * 2/4]{pictures/Diagramm_salazar.png}
	\caption{Diagramm zur Luftdruckberechnung}
	\label{pic:Diagramm_Salazar}
\end{figure}
\begin{equation}
	p(r, \theta, t) = j\rho_{0}c\frac{U_{0}}{\lambda}\int_{S}\frac{1}{r'}e^{j(\omega t-kr')}dS
	\label{equ_pressure}
\end{equation}
mit\\$r$ = Distanz eines Punktes zum Membranenmittelpunkt\\$r'$ = Distanz eines Punktes zur Membranenoberfläche\\$p_{0}$ = Nominaldurck\\$c$ = Ausbreitungsgeschwindigkeit der Luft\\$S$ = Membranenoberfläche\\$U_{0}$ = Geschwindigkeit der Membranenoberfläche\\$\lambda$ = Wellenlänge\\$t$ = Zeit\\$k$ = Kreiswellenzahl\\Aus: \cite{Salazar_Line_array_Thesis}\\Für eine einfache Visualisierung spielten die absoluten Werte keine Rolle, da nur die Verteilung dargestellt werden musste. Zudem wurde das Verhalten bei einer konstanten Frequenz simuliert. Somit konnten alle Vorfaktoren weggelassen werden. Wenn ein Line-Array als Aufreihung von Punktschallquellen angesehen werden kann, verschwindet zudem auch der Unterschied zwischen r und r'.\\In MATLAB R2023b wurde die Simulation berechnet. Dazu wurde die bekannte Länge von 121cm in Punkte mit 1cm Distanz unterteilt (also $N = 121$) und jeder Punkt als ideale Punktschallquelle angesehen. Nun wurde diese Punktlinie in sechs Abschnitte eingeteilt und allen Punkten innerhalb eines Abschnitts die gleiche Verzögerung und Verstärkung zugewiesen. Um den (relativen) Luftdruck an einem Punkt im Raum zu berechnen, kann nun der Beitrag jeder Quelle mit Gleichung \ref{eq:pressure_calculation} als komplexe Zahl berechnet und summiert werden. Somit konnten die Auslöschungen durch Phasenverschiebungen abgebildet werden. Zur besseren Darstellung kann das Ergebnis noch in dB umgerechnet werden.
\begin{equation}
	p_{dB}(t,r,\omega, k) = 10\cdot log_{10}( \abs{\sum_{1}^{N}a_{n}\frac{e^{j(\omega t_{delay} - r_{n} k)}}{r_{n}}})
	\label{eq:pressure_calculation}
\end{equation}
mit\\$r_{n}$ = Distanz eines Punktes zur Quelle\\$t_{delay}$ = Verzögerungszeit pro Line-Array Abschnitt\\$a_{n}$ = Verstärkung des Line-Array-Abschnitts\\$k$ = Kreiswellenzahl\\Zur visuellen Darstellung wurde ein Feld aus 600x600 Punkten definiert und das Line-Array links mittig an den Rand platziert. Zunächst wurde der Gain für alle Elemente auf 1 gesetzt. Abbildungen \ref{pic:linearray_simu_500Hz} bis \ref{pic:linearray_simu_9000Hz} zeigen diese Simulation bei verschiedenen Frequenzen und Phasenverschiebungen (\Delta Phi) zwischen den Elementen. Deutlich zu sehen der Übergang von omnidirektionaler zu stark direktionaler Abstrahlung. Jedoch hat die Phasenverschiebung bei zunehmender Frequenz immer weniger Einfluss. Ausserdem sind \textit{Side-Lobes\footnote{Abstrahlungen ausserhalb der Haupt-Achse. Meistens sind diese unerwünscht.}} deutlich sichtbar (siehe \ref{pic:linearray_simu_2500Hz_dphi100}).
\newpage
\begin{figure}[H]
	\centering
	\begin{subfigure}{\textwidth/3-3mm}
		\centering
		\includegraphics[width=\textwidth-2mm]{pictures/Matlab_Simulation_500Hz_dPhi0.png}
		\caption{\Delta Phi = 0\textdegree}
		\label{pic:linearray_simu_500Hz_dphi0}
		\vspace{2mm}
	\end{subfigure}
	\begin{subfigure}{\textwidth/3-3mm}
		\centering
		\includegraphics[width=\textwidth-2mm]{pictures/Matlab_Simulation_500Hz_dPhi50.png}
		\caption{\Delta Phi = 50\textdegree}
		\label{pic:linearray_simu_500Hz_dphi50}
		\vspace{2mm}
	\end{subfigure}
	\begin{subfigure}{\textwidth/3-3mm}
		\centering
		\includegraphics[width=\textwidth-2mm]{pictures/Matlab_Simulation_500Hz_dPhi100.png}
		\caption{\Delta Phi = 100\textdegree}
		\label{pic:linearray_simu_500Hz_dphi100}
		\vspace{2mm}
	\end{subfigure}
	\caption{Simulation bei 500 Hz}
	\label{pic:linearray_simu_500Hz}
\end{figure}
\begin{figure}[H]
	\centering
	\begin{subfigure}{\textwidth/3-3mm}
		\centering
		\includegraphics[width=\textwidth-2mm]{pictures/Matlab_Simulation_1500Hz_dPhi0.png}
		\caption{\Delta Phi = 0\textdegree}
		\label{pic:linearray_simu_1500Hz_dphi0}
		\vspace{2mm}
	\end{subfigure}
	\begin{subfigure}{\textwidth/3-3mm}
		\centering
		\includegraphics[width=\textwidth-2mm]{pictures/Matlab_Simulation_1500Hz_dPhi50.png}
		\caption{\Delta Phi = 50\textdegree}
		\label{pic:linearray_simu_1500Hz_dphi50}
		\vspace{2mm}
	\end{subfigure}
	\begin{subfigure}{\textwidth/3-3mm}
		\centering
		\includegraphics[width=\textwidth-2mm]{pictures/Matlab_Simulation_1500Hz_dPhi100.png}
		\caption{\Delta Phi = 100\textdegree}
		\label{pic:linearray_simu_1500Hz_dphi100}
		\vspace{2mm}
	\end{subfigure}
	\caption{Simulation bei 1500 Hz}
	\label{pic:linearray_simu_1500Hz}
\end{figure}
\begin{figure}[H]
	\centering
	\begin{subfigure}{\textwidth/3-3mm}
		\centering
		\includegraphics[width=\textwidth-2mm]{pictures/Matlab_Simulation_2500Hz_dPhi0.png}
		\caption{\Delta Phi = 0\textdegree}
		\label{pic:linearray_simu_2500Hz_dphi0}
		\vspace{2mm}
	\end{subfigure}
	\begin{subfigure}{\textwidth/3-3mm}
		\centering
		\includegraphics[width=\textwidth-2mm]{pictures/Matlab_Simulation_2500Hz_dPhi50.png}
		\caption{\Delta Phi = 50\textdegree}
		\label{pic:linearray_simu_2500Hz_dphi50}
		\vspace{2mm}
	\end{subfigure}
	\begin{subfigure}{\textwidth/3-3mm}
		\centering
		\includegraphics[width=\textwidth-2mm]{pictures/Matlab_Simulation_2500Hz_dPhi100.png}
		\caption{\Delta Phi = 100\textdegree}
		\label{pic:linearray_simu_2500Hz_dphi100}
		\vspace{2mm}
	\end{subfigure}
	\caption{Simulation bei 2500 Hz}
	\label{pic:linearray_simu_2500Hz}
\end{figure}
\begin{figure}[H]
	\centering
	\begin{subfigure}{\textwidth/3-3mm}
		\centering
		\includegraphics[width=\textwidth-2mm]{pictures/Matlab_Simulation_9000Hz_dPhi0.png}
		\caption{\Delta Phi = 0\textdegree}
		\label{pic:linearray_simu_9000Hz_dphi0}
		\vspace{2mm}
	\end{subfigure}
	\begin{subfigure}{\textwidth/3-3mm}
		\centering
		\includegraphics[width=\textwidth-2mm]{pictures/Matlab_Simulation_9000Hz_dPhi50.png}
		\caption{\Delta Phi = 50\textdegree}
		\label{pic:linearray_simu_9000Hz_dphi50}
		\vspace{2mm}
	\end{subfigure}
	\begin{subfigure}{\textwidth/3-3mm}
		\centering
		\includegraphics[width=\textwidth-2mm]{pictures/Matlab_Simulation_9000Hz_dPhi100.png}
		\caption{\Delta Phi = 100\textdegree}
		\label{pic:linearray_simu_9000Hz_dphi100}
		\vspace{2mm}
	\end{subfigure}
	\caption{Simulation bei 9000 Hz}
	\label{pic:linearray_simu_9000Hz}
\end{figure}\newpage
\noindent Diese Simulation konnte nun als Grundlage dienen, um die Verzögerungszeiten (\textit{delays}) jedes Arrayelements zu berechnen. Zudem könnten die Side-Lobes durch Tiefpass-Filter an den Enden weiter reduziert werden.
\begin{quote}
	The upper limit of a vertical array’s pattern control is always set by the inter-driver spacing: the design
	challenge is to minimize this dimension while optimizing frequency response and maximum output, and do it without imposing excessive cost. Line arrays become increasingly directional as frequency increases: at high frequencies they are too directional to be acoustically useful. However, if we have individual DSP available for each driver, we can use it to make the array acoustically “shorter” as frequency increases – this will keep the vertical directivity more consistent.\\Aus: \cite{R-H_LineArrayTheory}
\end{quote}
Die Simulation geht an dieser Stelle davon aus, dass sich ein Element 100\% gleichmässig und ohne Reflektionen bewegt. Zudem wird der Einfluss des Gehäuses vernachlässigt. Somit stellt diese Berechnung eine reine Veranschaulichung dar.\\Ein weiterer Punkt, welcher auch schon in \cite{R-H_LineArrayTheory} beschrieben wird, ist, dass durch die Gehäusedimensionen dem Beam-steering eine physikalische Grenze\footnote{Idealerweise würde ein Line-Array undendlich lang sein und aus unendlich kleinen Punktquellen bestehen. Dies ist jedoch schwierig herzustellen.} gesetzt wird:
\begin{itemize}
	\item Um genügend Direktionalität zu erreichen, muss das Line Array mindestens \textbf{$2\lambda$} lang sein.
	\item Um keine Side-Lobes zu haben, dürfen die Elemente maximal \textbf{$\frac{\lambda}{2}$} voneinander entfernt sein.
\end{itemize} 
Mit den bekannten Dimensionen können somit die Grenzen berechnet werden.
\begin{multline}
	f_{min} = \frac{c_{sound}}{\frac{l_{array}}{2}} \rightarrow \frac{343.5\frac{m}{s}}{\frac{1.21m}{2}} = \textbf{567.8Hz}\newline
	f_{max} = \frac{c_{sound}}{2\cdot l_{element}} \rightarrow \frac{343.5\frac{m}{s}}{2\cdot 0.202m} = \textbf{850.2Hz}
\end{multline}
Dieser Frequenzbereich ist natürlich sehr eng. Durch mehr Elemente könnte dieser auf höhere Frequenzen erweitert werden. Im Rahmen dieser Arbeit wird jedoch darauf verzichtet.